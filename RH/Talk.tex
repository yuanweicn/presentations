\documentclass[11pt]{amsart}
\usepackage{geometry}                % See geometry.pdf to learn the layout options. There are lots.
\geometry{letterpaper}                   % ... or a4paper or a5paper or ... 
%\geometry{landscape}                % Activate for for rotated page geometry
%\usepackage[parfill]{parskip}    % Activate to begin paragraphs with an empty line rather than an indent
\usepackage{amssymb}
\usepackage{latexsym}
\usepackage{amsmath}
\usepackage{amsfonts}
\usepackage{graphicx}
\usepackage{epstopdf}
\usepackage{color} %% to use color: \textcolor{color}{words to be in color} 
\DeclareGraphicsRule{.tif}{png}{.png}{`convert #1 `dirname #1`/`basename #1 .tif`.png}


\newcommand{\C}{\mathbb C }
\newcommand{\R}{\mathbb R}
\newcommand{\Z}{\mathbb Z}
\newcommand{\G}{\mathcal G}

\title{Minor talk}
\author{Wei Yuan}                                       

\begin{document}
\maketitle

Thank you all for coming. The title of my talk is "A Necessary and Sufficient Condition for 
Riemann Hypothesis".  Whenever I think about analytic number theory, I will connect it with pages 
after pages of complicated computation and estimation, since I am not good at calculating, 
analytic number theory is always a mystery to me. Finally this term, I make up my mind to do some 
computation and hope that I can at least decipher a little piece of their terminology and methodology. 
I can not think out best way to achieve this by playing with the $\zeta$ function, at least there are 
so many people obsessed with it. 

I believe everyone here know what the Riemann's Zeta function is. But because review always help 
us understand a subject better, I will start by reviewing some basic facts about $\zeta$ function. Through
this talk, we use $z$ to denote complex number, and $x$ as its real part, $y$ as its maginary part, that is
$z = x + iy$, where $i$ is the  imaginary unit. Before talk about $\zeta$ function, we need first remember 
some property of $\Gamma$ function, it's one of the key ingredients to study $\zeta$ function(extend it
to the entire complex plane except $1$). 

Define the $\Gamma$ function by:
\begin{align*}
\Gamma(z) = \int^{\infty}_{0} e^{-t} t^{z-1} dt.
\end{align*}
This defines an analytic function on $x > 0$. $\Gamma$ can be continued to a nowhere-zero meromorphic
function on all $\mathbb{C}$ with simple ploes at $0, -1, -2, \ldots$. Equivalently, $\Gamma(1-z)$ has simple
poles at $z= 1, 2, \ldots$, all the positive integers, and the residue of $\Gamma(1-z)$ at $z = k$ is :
\begin{align*}
\frac{(-1)^{k-1}}{(k-1)!}.
\end{align*}
We will need this fact latter.

With these in mind, we can start talking about Riemann's zeta-function. $\zeta(z)$ has its origin in the following 
identity:
\begin{align*}
\sum^{\infty}_{n=1} \frac{1}{n^z} = \zeta(z) = \prod_{p} (1 - \frac{1}{p^z})^{-1}, x > 1
\end{align*}
The left hand side is a  Dirichlet series, the right hand side is the Eular product(infinite product).(Talk about the upper
bound of $\zeta$).

By the equation above, it's not hard to get the following expression of  $\frac{1}{\zeta(z)}$ for $x > 1$:
\begin{align*}
\frac{1}{\zeta(z)} = \prod_{p} (1 - \frac{1}{p^z})  
                                  = \sum^{\infty}_{n=1} \frac{\mu(n)}{n^z},  x > 1.
\end{align*}
(Explain the m$\ddot{o}$bius function and the upper bound of $\frac{1}{\zeta(z)}$.)

But what exactly does the zeta function looks like? Thanks  to the modern technology---show the pics. We 
can have a vague picture of what the landscape of the zeta function looks like.

This picture depict the absolute value of zeta function $0 < x <  2$.
Despite the infinite peak over the number $1$,  it seems possible to extend the landscape of zeta function 
to the left of the line passing through number $1$. Although the former equation may not make sense if you 
put in numbers with real part less than $1$, the geometry of the landscape indicate otherwise.
Since the landscapes has an extraordinarily rigid geometry, there is only one way that the landscpe could be 
expended. In his paper "On the number of primes less than a given magnitued(1859)", Riemann succeeded in finding 
another formula that could be used to build the missing landscape and gave the following function equation:
\begin{align*}
\zeta(z) = 2^{z}\pi^{z-1}sin(\frac{1}{2}z\pi)\Gamma(1-z)\zeta(1-z),  \forall z \in \C.
\end{align*}
By this identity, we can see that since $-2, -4, -6, \ldots$, are the zeros of $\sin(\frac{1}{2}\pi z)$, $\zeta$ has 
zeros at those points(explain the behavior of $\Gamma$ and $\zeta$ when $z > 1$). Because those zeros 
are so easy to find, there are called the trivial zeros.

If define 
\begin{align*}
\xi(z) &= \frac{1}{2}z(z-1)\pi^{-\frac{1}{2}z}\Gamma(\frac{1}{2}z)\zeta(z), \\
\Xi(z) &= \xi(\frac{1}{2} + iz).
\end{align*}
The functional equation can be write in a symmetric form:$\Xi(z) = \Xi(-z)$, in another word $\Xi(z)$ is a even
function. Notice that $\Xi(z)$ is an entire function and do not have zero outside the "critical strip" ( $ 0 \leq x \leq 1$), since the zeros at 
negative even number are cancel by the simple poles of $\Gamma (\frac{1}{2}z)$, and $z \in \C $ is a 
nontrivial zero of $\zeta$ iff $z$ is a zero of $\Xi(z)$.

In the very same paper, Riemann states the \textbf{Riemann hypothesis} in terms
of roots of $\Xi(z)$(actually is $\frac{\Xi(z)}{z(1-z)}$) :\\
\textbf{......it is very probable that all roots are real. Of course one would wish for a stricter proof here; 
I have for the time being, after some fleeting futile attempts, provisionally put aside the search for this, 
as it appears unnecessary for the next objective of my investigation.}

\begin{center}
\textcolor{red}{ \Large \textbf{Riemann hypothesis}}\\
\textcolor{red}{\Large All (nontrival) zero of $\zeta(z)$ have real part $\frac{1}{2}$.}
\end{center}
Since conjectured by Riemann, it has withstood concentrated efforts from many outstanding mathematicians for
almost 150 years, and is considered by many mathematicians to be the most important unresolved problem in
pure mathematics.

For the rest of this talk, I will give a Necessary and Sufficient Conditions for the Riemann Hypothesis
which is a refinement of a classical result due to M. Riesz(1916). In order to state the condition, we
first introduce the Bernoulli's number $B_k$, those are the numbers given by:
\begin{align*}
\frac{z}{e^{z} -1} + \frac{1}{2}z =  1 + B_1\frac{z^{2}}{2!} - B_2 \frac{z^{4}}{4!} + \cdots. \\
\end{align*}
and it can be showed that 
\begin{align*}
\zeta(2k) = 2^{2k -1}\pi^{2k}\frac{B_k}{(2k)!}(k = 1, 2, \ldots). 
\end{align*}
Now, let 
\begin{align*}
A_k &= \frac{(2\pi)^{2}}{(2k+2)(2k+1)} \frac{B_{k+1}}{B_k} = \frac{\zeta(2m+2)}{\zeta(2m)}.\\
f(t) &= \sum^{\infty}_{k = 1} \frac{(-1)^{k-1}A_k}{(k-1)!}t^{k}.
\end{align*}
The condition can be stated as following:\\
The Riemann Hypothesis is equivalent to the following statement:\\

For any $\epsilon > 0 $, there exists a constant $C(\epsilon) > 0$ such that for any $t > 1$, 
%actually only need a a \in R, t >a the following hold
\begin{align*}
|f(t)| \leq C(\epsilon) t^{\frac{1}{4} + \epsilon}  .
\end{align*}
To see why this is true, we first show that RH implies our upper bound estimate.\\
Let 
\begin{align*}
I(c,t) = \frac{1}{2\pi i} \int^{c + \infty}_{c -\infty} \frac{\zeta(2z +2)\Gamma(1-z)}{\zeta(2z)}t^{z} dz.
\end{align*}
for all $c \in \R$ and $c \notin \Z$.























\end{document}  